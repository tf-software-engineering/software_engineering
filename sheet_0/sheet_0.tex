\documentclass{scrartcl}
\usepackage{german}
\usepackage[utf8]{inputenc}
\usepackage[german]{babel}

\newcommand{\RNum}[1]{\uppercase\expandafter{\romannumeral #1\relax}}

% zusätzliche mathematische Symbole, AMS=American Mathematical Society 
\usepackage{amssymb}
\usepackage{amsmath}

% fürs Einbinden von Graphiken
\usepackage{graphicx}

% für Namen etc. in Kopf- oder Fußzeile
\usepackage{fancyhdr}

% erlaubt benutzerdefinierte Kopfzeilen 
\pagestyle{fancy}

% Definition der Kopfzeile
\iffalse
\lhead{
\begin{tabular}{ll}
Nils Hagner & 4346038 \\
\end{tabular}
}
\fi
\chead{}
\rhead{\today{}}
\lfoot{}
\cfoot{Seite \thepage}
\rfoot{} 

\begin{document}

\section*{Solutions to Excercise Sheet 0}

\subsection*{Exercise 1 – Are Software Engineers Managers?}

\subsection*{Exercise 2 – Does Software Reliability Matter?}

The heartbleed bug

The heartbleed bug was a vulnerability introduced to the Codebase of OpenSSl in December 2011. Since March 2012, affected Versions of OpenSSL allowed atackers to extract information from RAM which was often used before to store private keys and other important information.

The Vulnerability was caused by the implementation of a the 'heartbeat', which is usually used to determine wether a TSL-encrypted connection to a server is still usable and the other partner is responding. This should have been done by sending a message and getting the same content of it back, but the message 

The damage which was caused is not excatly known, because


Provide a general description of the case, followed by a more detailed description of the software-
related issue and its consequences.
 Quantify the damages caused and argue why the case is relevant.
Discuss in how far the incident (following official reports, or in your opinion) is related to issues with
requirements, design, quality assurance, management, or usage under specified conditions.

\section*{Survey}

\subsection*{1. Expectations}
Our expetcations in Softwaretechnik course are:
\begin{itemize}
\item to understand the process of software production
\item Learning more about General Planning as well as
Specific to Software Development
\item to gain knowledge in requirement analysis, process scheduling,
organisation, resource distribution, design and testing
\item to learn how to use basic and maybe some advanced techniques,
models and patterns in software development
\item learn the modern techniques of software engineering (agil instead of waterfall), e.g. DDD, TDD, BDD
\item getting to know valuable Design-Patterns
\item heavy focus on software architecture (more than just some design patterns)
\item to focus deeper on software architecture in General
\item get smarter
\item have fun
\item Thinking about Software in a more General way
\item Getting tools for Attacking Problems with the help of Software
\end{itemize}

We think that this will help us with our carreer because

\begin{itemize}
	\item managing is important, even if one is not 
\end{itemize}
\subsection*{2. Previous Experience}


\begin{tabular}{| p{10cm} | c | c | c | c | c | c | c | c | c | c | c |}
	\hline
	& 0& 1& 2& 3& 4& 5& 6& 7& 8& 9& 10\\
	\hline
	Project Management (cf. Exercise 1) \\
	\hline
	Nils Hagner &x&&&&&&&&&&\\ \hline
	Michael Fleig &&&x&&&&&&&&\\ \hline
	Anush Davtyan &&&x&&&&&&&&\\ \hline
	Felix Karg &&&&&&&x&&&&\\ \hline
	Requirements Engineering (capturing and managing requirements from users or clients)\\
	\hline
	Nils Hagner &x&&&&&&&&&&\\ \hline
	Michael Fleig &&&&&&x&&&&&\\ \hline
	Anush Davtyan &&&&&x&&&&&&\\ \hline
	Felix Karg &&&&&x&&&&&&\\ \hline
	Programming (writing code, fixing bugs)\\
	\hline
	Nils Hagner &&x&&&&&&&&&\\ \hline
	Michael Fleig &&&&&&&&&&x&\\ \hline
	Anush Davtyan &&&&&x&&&&&&\\ \hline
	Felix Karg &&&&&&&&&&x&\\ \hline
	Design Modelling (creating an architecture or behavior model of a solution)\\
	\hline
	Nils Hagner &x&&&&&&&&&&\\ \hline
	Michael Fleig &&&&&&&&&&&x\\ \hline
	Anush Davtyan &&&&x&&&&&&&\\ \hline
	Felix Karg &&&&&&&&x&&&\\ \hline
	Software Quality Assurance (e.g., testing, code review, formal verification)\\
	\hline
	Nils Hagner &x&&&&&&&&&&\\ \hline
	Michael Fleig &&&&&&&&&x&&\\ \hline
	Anush Davtyan &&&&x&&&&&&&\\ \hline
	Felix Karg &&&&&&&&&x&&\\ \hline
\end{tabular}


\subsection*{3. Regarding the Softwarepraktikum...}
\begin{tabular} {| p{7cm} | c | c | c | c |}
	\hline
	&Nils Hagner&Michael Fleig&Anush Davtyan&Felix Karg\\
	\hline
	I will be participating in it this semester. &&&&\\
	\hline
	I have already taken part. &&&&\\
	\hline
	I will participate in it in the following semesters. &x&x&x&x\\
	\hline
	It is not part of my study plan. &&&&\\
	\hline
\end{tabular}

\subsection*{4. Language}
$\square$ German.\\
$\square$ I prefer German, but English is okay.\\
$\boxtimes$ I prefer English, but German is okay.\\
$\square$ English.\\
\end{document}
