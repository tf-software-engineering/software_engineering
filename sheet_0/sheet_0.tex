\documentclass{scrartcl}
\usepackage[bottom=10em]{geometry}
\usepackage{german}
\usepackage[utf8]{inputenc}
\usepackage[german]{babel}

\newcommand{\RNum}[1]{\uppercase\expandafter{\romannumeral #1\relax}}

% zusätzliche mathematische Symbole, AMS=American Mathematical Society
\usepackage{amssymb}
\usepackage{amsmath}

% fürs Einbinden von Graphiken
\usepackage{graphicx}

% für Namen etc. in Kopf- oder Fußzeile
\usepackage{fancyhdr}

% erlaubt benutzerdefinierte Kopfzeilen
\pagestyle{fancy}

% Definition der Kopfzeile
\iffalse
\lhead{
\begin{tabular}{ll}
Nils Hagner & 4346038 \\
Felix Karg & 4342014 \\
\end{tabular}
}
\fi
\chead{}
\rhead{\today{}}
\lfoot{}
\cfoot{Seite \thepage}
\rfoot{}

\begin{document}

\section*{Solutions to Exercise Sheet 0}

\subsection*{Exercise 1 – Are Software Engineers Managers?}
\begin{itemize}
\item The software development is almost always done in a team.
Effective teamwork can bring great results, but it needs to be well organized.
The communication between the members of a group, a proper distribution of work,
setting priorities or time scheduling are responsibilities of a project manager.

Why is the work distribution important?
A proper work distribution can increase the productivity and improve the quality of out coming product.
There are few common mistakes that can be done:
\begin{itemize}
\item not knowing or not paying attention to the abilities and key skills of team members.
This could lead to time waste and frustration among the developers.
\item On the other hand not providing the less experienced employees new challenges.
A good manager should give the team members a possibility to develop themselves,
otherwise people can lose the interest in work.
\item To give too much work to the most productive, most experienced or most responsible member of a team.
An overworked employee would not do his best, would feel stressed and used, and it would spread the rest of the team.
As we see the work distribution can be a challenging task that requires not only management or executive skills,
but also understanding of the capabilities and psychology of team members.
\end{itemize}
This problem is not specific to software development,
however it is applicable and very important for people working in IT.

Why 'normal' managers couldn't do this:
Experience in managing would be beneficial to this, but in software development it is especially critical
to have at least some knowledge of the tasks one distributes to others. For example it can be harmful to
the development process if managers don't understand the real value of testing or underestimate the cost
of implementing a feature.

\item Managing the interaction between different Software components (eg. libraries etc.)
Defining the project boundaries can be a challenging task.
For software development it is essential to determine the standards of the work,
APIs, code quality, architecture and to manage resources(e.g. time, money, RAM, CPU-Power).
This has to be done by a project manager. However this management activity requires knowledges
and experience in computer science. Technical limitations has to be considered.
Unreasonable time demands for example or money limitations can lead to degradation of the quality
or in the worst case to not accomplishing the task, which is waste of time and money.

\item Good software reflects to domain it tries to solve.
Most software is developed for other industries than our own. So developer have to understand the domain they are working in. We can't build a banking software without understanding how accounting works.
To achieve this communication with domain experts is essential.
Although translation between different teams/ departments is a typical management activity in software development it requires a requires a thorough understanding of our industry.
\end{itemize}

\subsection*{Exercise 2 – Does Software Reliability Matter?}

\subsubsection*{The heartbleed vulnerability}

The heartbleed bug was a vulnerability introduced to the Codebase of OpenSSl in December 2011. Since March 2012, affected Versions of OpenSSL allowed atackers to extract information from the RAM of connected servers which was often used before to store private keys and other important information.

The vulnerability was caused by the implementation of a the 'heartbeat', which is usually used to determine wether a TSL-encrypted connection to a server is still usable and the other partner is responding. This should have been done by sending a message and getting the same content of it back, but the message could be altered in a way which would lead the receiving system to send up to 64 KiB unused memory following the buffer which was used to store the received information. Because this memory was likely to be used by OpenSSL before, it often contained sensible information.\\

The exact damage which was caused is not exactly known, because the bug was not noticed for nearly two years, but there were some incidents where
Social Insurance Numbers from about 900 Canadian Taxpayers were stolen, and one where researchers could extract their own private keys from a machine.\\
By now, it is thought of one of the most threatening vulnerabilities in the last few years, and one which has had a great media coverage.
The bug could have been noticed way earlier:\\
Firstly, the bug in the implementation could have been noticed by the author.
Secondly, if the implementation which was adopted into the codebase would have been checked better, the vulnerability would not have existed.
And thirdly: If the code would have been refactored within the two years, one might have noticed the flaw.

So it was mainly bad Quality assurance which led to the adoption of the vulnerable implementation into the codebase, with more attention and help it may could have been prevented.
\em Sources: http://heartbleed.com/ \hspace{1cm} https://en.wikipedia.org/wiki/Heartbleed
\em 
\section*{Survey}

\subsection*{1. Expectations}
Our expectations for the Softwaretechnik/Software Engineering course are:
\begin{itemize}
\item Understanding the procedure of software production, including common mishaps at each step
\item Learning more about the planning process in general as well as specifically applied to Software Development
\item to gain knowledge in requirement analysis, process scheduling (..), organization, resource distribution, design and testing
\item to learn how to use basic and maybe some advanced techniques, models and patterns in software development
\item learn the modern techniques of software engineering (agil instead of waterfall), e.g. D(omain)D(riven)D(esign), T(est)D(riven)D(esign), B(ehaviour)D(riven)D(esign)
\item getting to know valuable Design-Patterns and being able to apply them more generally
\item to see a focus on software architecture (more than just some design patterns, e.g. scalability, ...) in general, specifically how to design and maybe use them
\item to get smarter, have some fun, and learn a lot along the way, not only for the further studying or working but also for life
\item think about Software in a more General way, as in having a deeper knowledge about it's usual background
\item getting tools (roughly specific ideas) for attacking Problems with the help of Software, or simply training that
\end{itemize}

We think that this will help us with our career because management experience (even if only in theory) can help in any job additionally to it being useful for Life on a more general scale.
As upcoming Software Engineers it's more than useful having at least some basic knowledge about not only how to code, or rather, in the actual field of Software Engineering.
Additionally, hobby projects or help on Open-Source Projects in spare time could be done more efficiently, knowing some of the involved processes, usually used structures for more efficiancy and more.
Getting better at problem solving is another one, working as a team and having fun just makes studying (and later on working) a whole lot easier.

\subsection*{2. Previous Experience}


\begin{tabular}{| p{10cm} | c | c | c | c | c | c | c | c | c | c | c |}
	\hline
	& 0& 1& 2& 3& 4& 5& 6& 7& 8& 9& 10\\
	\hline
	Project Management (cf. Exercise 1) \\
	\hline
	Nils Hagner &x&&&&&&&&&&\\ \hline
	Michael Fleig &&&x&&&&&&&&\\ \hline
	Anush Davtyan &&&x&&&&&&&&\\ \hline
	Felix Karg &&&&&&&x&&&&\\ \hline
	Requirements Engineering (capturing and managing requirements from users or clients)\\
	\hline
	Nils Hagner &x&&&&&&&&&&\\ \hline
	Michael Fleig &&&&&&x&&&&&\\ \hline
	Anush Davtyan &&&&&x&&&&&&\\ \hline
	Felix Karg &&&&&x&&&&&&\\ \hline
	Programming (writing code, fixing bugs)\\
	\hline
	Nils Hagner &&x&&&&&&&&&\\ \hline
	Michael Fleig &&&&&&&&&&x&\\ \hline
	Anush Davtyan &&&&&x&&&&&&\\ \hline
	Felix Karg &&&&&&&&&&x&\\ \hline
	Design Modelling (creating an architecture or behavior model of a solution)\\
	\hline
	Nils Hagner &x&&&&&&&&&&\\ \hline
	Michael Fleig &&&&&&&&&&&x\\ \hline
	Anush Davtyan &&&&x&&&&&&&\\ \hline
	Felix Karg &&&&&&&&x&&&\\ \hline
	Software Quality Assurance (e.g., testing, code review, formal verification)\\
	\hline
	Nils Hagner &x&&&&&&&&&&\\ \hline
	Michael Fleig &&&&&&&&&x&&\\ \hline
	Anush Davtyan &&&&x&&&&&&&\\ \hline
	Felix Karg &&&&&&&&&x&&\\ \hline
\end{tabular}


\subsection*{3. Regarding the Softwarepraktikum...}
\begin{tabular} {| p{7cm} | c | c | c | c |}
	\hline
	&Nils Hagner&Michael Fleig&Anush Davtyan&Felix Karg\\
	\hline
	I will be participating in it this semester. &&&&\\
	\hline
	I have already taken part. &&&&\\
	\hline
	I will participate in it in the following semesters. &x&x&x&x\\
	\hline
	It is not part of my study plan. &&&&\\
	\hline
\end{tabular}

\subsection*{4. Language}
$\square$ German.\\
$\square$ I prefer German, but English is okay.\\
$\boxtimes$ I prefer English, but German is okay.\\
$\square$ English.\\
\end{document}
