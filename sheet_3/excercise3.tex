\documentclass{scrartcl}
\usepackage{german}
\usepackage{fontspec}
\setmainfont{Linux Libertine O}
%\usepackage[utf8]{inputenc}
\usepackage[german]{babel}

% zusätzliche mathematische Symbole, AMS=American Mathematical Society
\usepackage{amssymb}
\usepackage{amsmath}
% fürs Einbinden von Graphiken
\usepackage{graphicx}
\usepackage{verbatim}

% für Namen etc. in Kopf- oder Fußzeile
\usepackage{fancyhdr}

% um tabelle nach links zu schieben
\usepackage{changepage}

% erlaubt benutzerdefinierte Kopfzeilen
\pagestyle{fancy}

% Definition der Kopfzeile
\lhead{
\begin{tabular}{lllll}
Nils Hagner & 4346038 & Felix Karg & 4342014 \\
Michael Fleig & 4340085 & Anush Davtyan &4368689
\end{tabular}
}
\chead{}
\rhead{\today{}}
\lfoot{}
\cfoot{Seite \thepage}
\rfoot{}

\begin{document}

\section*{Solutions for Excercise sheet 3}
\subsection*{Exercise 3 – Creation of Decision Tables}

\begin{itemize}
    \item[i]
    1. For small packages, the shipping costs depend on the weight of the items in the shopping cart, there is a fixed price for the first 2kg and a variable fee for each additional kg:\\
    \begin{tabular} {l l l}
    	Type & First kg. &     Additional kg.\\
    	Metropolitan &     3.00 &     1\\
    	Intermediate &     5 &     1.5\\
    	Rural &     10.00 &     2.5
    \end{tabular}\\\\
    2. The parcel shipping costs for the first kilogram and additional kilograms are given on the following
    table:\\
    \begin{tabular} {l l l}
	Type & First kg. &     Additional kg.\\
	Metropolitan &     1.00 &     0.75\\
	Intermediate &     2.25 &     1.25\\
	Rural &     5.00 &     2.75
    \end{tabular}\\
    3. If the shipping address is in the same city as the online shop, a charge on delivery (COD) shipping option should be offered, for a fixed price of 10 Euro.\\
    4. There is a special offer: For rural areas, small but heavy packages (volumetric weight less than 5kg but more than 5kg actual weight) pay the price of intermediate cities.\\
    \begin{adjustwidth}{-4em}{-4em}
    \begin{tabular}{|p{10em} || p{2.5em} | p{3em} | p{3em} | p{3em} | p{1em} | p{3em} | p{3em} | p{3em} | p{3em} | p{4em} |}
    	\hline
    	DT: Price calculations & r1 (1.1)& r2 (1.2)& r3 (1.3)& r4 (4)& r5 (3)& r6 (2.1)& r7 (2.2)& r8 (2.3)\\
    	\hline \hline
    	effective weight $\leq$ 5kg &x&x&x&x&*&-&-&-\\
    	\hline
    	effective weight $>$ 5kg &-&-&-&-&*&x&x&x\\
    	\hline
    	metropolitan&x&-&-&-&*&x&-&-\\
    	\hline
    	intermediate&-&x&-&-&*&-&x&-\\
    	\hline
    	rural&-&-&x&x&*&-&-&x\\
    	\hline
    	same city as shop&*&*&*&*&x&*&*&*\\
    	\hline
    	actual weight $>$ 5kg &*&*&-&x&*&*&*&*\\
    	\hline
    	\hline
    	display COD option &-&-&-&-&x&-&-&-\\
    	\hline
    	price calculation &3+w -1&2.25 +1.25w -1.25 &5 +2.75w -2.75&2.25 +1.25w -1.25&-&1 +0.75w -0.75&2.25 +1.25w -1.25&5 +2.75w -2.75\\
    	\hline 
    \end{tabular}
	\end{adjustwidth}
conflict axioms:\\
The adress is either metropolitan, intermediate or rural, so no other combination (e.g $metropolitan \land rural$) cannot happen:\\
\[
\varphi_{confl} = ¬(metropolitan \oplus rural \oplus intermediate) \Leftrightarrow ¬(c3 \oplus c4 \oplus c5) %\equiv?
\]
The effective weight can be exclusively either less than or more than 5 kg:\\
\[
\psi_{confl} = (c1 \land c2 )\lor(¬c1 \land ¬c2)
\]
    \item[ii]
    shipping to the \textbf{rural} area Niederaichbach, (so \textbf{not in the same city} where the shop is located) with the dimensions 29.7cm × 21cm × 20cm and a weight of 6.25kg gives us an effective weight of 5,9896 kg (\textbf{> 5 kg}) (a small package).\\
    So we can only use the rule \textbf{r4}, which give us a price of $2.25\cdot 1.25 \cdot 5.9896 -1.25\approx8.49€$
    \item[iii]
    The table is consistent because the interpreting function is a tautology, it has a rule for every possible combination of conditions:\\
    \begin{align*}
    &c1 \land ¬c2 \land c3 \land ¬c4 \land ¬c5 \land true \land true\\
    &\lor c1 \land ¬c2 \land ¬c3 \land c4 \land ¬c5 \land true \land true\\
    &\lor c1 \land ¬c2 \land ¬c3 \land ¬c4 \land c5 \land true \land ¬c7\\
    &\lor c1 \land ¬c2 \land ¬c3 \land ¬c4 \land c5 \land true \land c7\\
    &\lor ¬c1 \land c2 \land c3 \land ¬c4 \land ¬c5 \land true \land true\\
    &\lor ¬c1 \land c2 \land ¬c3 \land c4 \land ¬c5 \land true \land true\\
    &\lor ¬c1 \land c2 \land ¬c3 \land ¬c4 \land c5 \land true \land true\\
    &\lor true \land true \land true \land true \land true \land c6 \land true\\
    &\lor ¬(c3 \oplus c4 \oplus c5)\\
    &\lor (c1 \land c2 )\lor(¬c1 \land ¬c2)\\\\
    % \equiv ?
    &\Leftrightarrow ¬c2 \land c3 \land ¬c4 \land ¬c5\\
    &\lor ¬c2 \land ¬c3 \land c4 \land ¬c5 \\
    &\lor ¬c2 \land ¬c3 \land ¬c4 \land c5 \land ¬c7\\
    &\lor ¬c2 \land ¬c3 \land ¬c4 \land c5 \land c7\\
    &\lor c2 \land c3 \land ¬c4 \land ¬c5 \\
    &\lor c2 \land ¬c3 \land c4 \land ¬c5 \\
    &\lor c2 \land ¬c3 \land ¬c4 \land c5 \\
    &\lor c6\\
    &\lor ¬(c3 \oplus c4 \oplus c5)\\
    &\lor (c1 \land c2 )\lor(¬c1 \land ¬c2)\\
    % \equiv ?
    \end{align*}
    \newpage
    \begin{align*}
    &\Leftrightarrow ¬c2 \land c3 \land ¬c4 \land ¬c5\\
    &\lor ¬c2 \land ¬c3 \land c4 \land ¬c5 \\
    &\lor ¬c2 \land ¬c3 \land ¬c4 \land c5 \land true\\
    &\lor c2 \land c3 \land ¬c4 \land ¬c5 \\
    &\lor c2 \land ¬c3 \land c4 \land ¬c5 \\
    &\lor c2 \land ¬c3 \land ¬c4 \land c5 \\
    &\lor c6\\
    &\lor ¬(c3 \oplus c4 \oplus c5)\\
    &\lor (c1 \land c2 )\lor(¬c1 \land ¬c2)\\\\
	% \equiv ?
	&\Leftrightarrow ¬c2 \land c3 \land ¬c4 \land ¬c5\\
	&\lor ¬c2 \land ¬c3 \land c4 \land ¬c5 \\
	&\lor ¬c2 \land ¬c3 \land ¬c4 \land c5 \\
	&\lor c2 \land c3 \land ¬c4 \land ¬c5 \\
	&\lor c2 \land ¬c3 \land c4 \land ¬c5 \\
	&\lor c2 \land ¬c3 \land ¬c4 \land c5 \\
	&\lor c6\\
	&\lor ¬(c3 \oplus c4 \oplus c5)\\
	&\lor (c1 \land c2 )\lor(¬c1 \land ¬c2)\\
    \end{align*}
    because of $\psi_{confl}$ (and common sense) we can conclude that $c1 = ¬c2$ and get rid of the first (or second) row in the table ($c1$ or $c2$ in the interpreting function). The third and the fourth line differ in only one variable, so can be reduced to one line without $c7$.
\end{itemize}

\end{document}