\documentclass{scrartcl}
\usepackage[bottom=10em]{geometry}
\usepackage{german}
\usepackage[utf8]{inputenc}
\usepackage[german]{babel}

\usepackage{enumerate} % for the roman enumeration
\usepackage{listings} % for using language-specific syntax highlighting
\usepackage{minted}

\newcommand{\RNum}[1]{\uppercase\expandafter{\romannumeral #1\relax}}

% zusätzliche mathematische Symbole, AMS=American Mathematical Society
\usepackage{amssymb}
\usepackage{amsmath}

% fürs Einbinden von Graphiken
\usepackage{graphicx}

% für Namen etc. in Kopf- oder Fußzeile
\usepackage{fancyhdr}

% erlaubt benutzerdefinierte Kopfzeilen
\pagestyle{fancy}

% Definition der Kopfzeile
\iffalse
\lhead{
\begin{tabular}{ll}
Nils Hagner & 4346038 \\
Felix Karg & 4342014 \\
Michael Fleig & \\
Anush Davtyan & \\
\end{tabular}
}
\fi
\chead{}
\rhead{\today{}}
\lfoot{}
\cfoot{Seite \thepage}
\rfoot{}

\begin{document}

\section*{Solutions to Exercise Sheet 1}

\section*{Exercise 1 - Metrics}

\subsection*{1.1 Lines of Code Metrics}

\begin{enumerate}[i]
    \item $LOC_{tot} = 74 $\\
          $ LOC_{ne} = 74 - 10 = 64 $\\
          $ LOC_{pars} = 64 - 15 = 49 $\\
    \item Example Haskell-Code for contrasting the given MyQuickSort.java: \\
        \inputminted[linenos]{Haskell}{MyQuickSort.hs}


        $LOC_{parsH} = 9 - 2 = 7$ \\
        So there is $LOC_{pars}$ with 49 as well as $LOC_{parsH}$ with 7 (Order of magnititude: $n\ vs\ n^2$!).
        These are obviously two entirely different Programs, yet they are semantically equivalent (in that they)
        offer an interface to a function capable of sorting a List of items with an algorithm of the Quicksort-category.
\end{enumerate}



\subsection*{1.2 Cyclomatic Complexity}







\end{document}
