\documentclass{scrartcl}
\usepackage[bottom=10em]{geometry}
\usepackage{german}
\usepackage[utf8]{inputenc}
\usepackage[german]{babel}

\usepackage{enumerate} % for the roman enumeration
\usepackage{minted} % for using language-specific syntax highlighting

\newcommand{\RNum}[1]{\uppercase\expandafter{\romannumeral #1\relax}}

% zusätzliche mathematische Symbole, AMS=American Mathematical Society
\usepackage{amssymb}
\usepackage{amsmath}

% fürs Einbinden von Graphiken
\usepackage{graphicx}

% für Namen etc. in Kopf- oder Fußzeile
\usepackage{fancyhdr}

% erlaubt benutzerdefinierte Kopfzeilen
\pagestyle{fancy}

% Definition der Kopfzeile
\iffalse
\lhead{
\begin{tabular}{ll}
Nils Hagner & 4346038 \\
Felix Karg & 4342014 \\
Michael Fleig & 4340085 \\
Anush Davtyan & \\
\end{tabular}
}
\fi
\chead{}
\rhead{\today{}}
\lfoot{}
\cfoot{Seite \thepage}
\rfoot{}

\begin{document}

\section*{Solutions to Exercise Sheet 1}

\section*{Exercise 1 - Metrics}

\subsection*{1.1 Lines of Code Metrics}

\begin{enumerate}[i]
    \item $LOC_{tot} = 74 $\\
          $ LOC_{ne} = 74 - 10 = 64 $\\
          $ LOC_{pars} = 64 - 15 = 49 $\\
    \item Example Haskell-Code for contrasting the given MyQuickSort.java: \\
        \inputminted[linenos]{Haskell}{MyQuickSort.hs}



        $LOC_{parsH} = 9 - 2 = 7$ \\
        So there is $LOC_{pars}$ with 49 as well as $LOC_{parsH}$ with 7 (Order of magnititude: $n^2\ vs\ n$!).
        These are obviously two entirely different Programs, yet they are semantically equivalent in that they
        offer an interface to a function capable of sorting a List of items with a Quicksort-Algorithm. \\ \\

        In this case the recognized Pattern is to use a library or tool requiring a lot of (hardcoded) configuration,
        where a simpler one would clearly suffice. That way you would have a lot of managing / organizing / configurating
        overhead, which can be overblown to the fullest if wanted (resulting in an LOC of literally any number you wish).

    \item Metrics: \\
        $ LOCtot = 178  $ \\
        $ LOCne = 136   $ \\
        $ LOCpars = 120 $ \\
        It basically configures all the GUI-Elements (like buttons etc.), their Positions, Sizes,
        how they should behave when resizing the window or when being clicked, and more. \\
        The Program itself is a simulator for AI-Ants to find paths in a generated Maze, the whole Project written in C++.


\end{enumerate}



\subsection*{1.2 Cyclomatic Complexity}
\begin{enumerate}
	\item $ p = 3 $ \\
		$ n = 13 $ \\
		$ e = 17 $ \\
		$ v(G) = 17 - 13 + 3 = 7 $  \\
                The CFG will be on the last page.
	
	\item Junction points in a CFG do not alter the cyclomatic complexity as it adds an edge for each node.
	
\end{enumerate}






\end{document}
